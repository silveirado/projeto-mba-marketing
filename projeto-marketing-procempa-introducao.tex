\chapter{Introdução}

No presente cenário de constante evolução tecnológica, recursos cada vez mais limitados e cidadãos melhor informados e com necessidades crescentes, a Tecnologia da Informação firmou-se como um papel de destaque no apoio do crescimento e resolução de problemas das grandes cidades pelo mundo.

Nos últimos anos a Companhia de Processamento de Dados do Município de Porto Alegre - Procempa - colocou em prática diversas estratégias com o fim de potencializar projetos de melhorias para o cidadão e otimização de processos para potencializar o investimento público e garantir a transparência em seus processo.

Por ser uma companhia inovadora desde seu nascimento, o capital intelectual presente atualmente tem propiciado a entrega constante e satisfatória de melhorias para a cidade. Na empresa a experiência de colaboradores de longa data foi mesclada com o dinamismo de novos empregados, aprovados em concurso público em 2014, criando um ambiente de mudanças e melhorias constantes.

O presente trabalho buscará demonstrar estratégias de disseminação das tecnologias utilizadas pela empresa para alavancar a companhia como um centro de referência em metodologia e construção de software e incentivo à iniciativas da Prefeitura de Porto Alegre no sentido de atrair novas empresas, fornecendo apoio as comunidades de software livre, bem como dados públicos para incentivar investimentos e aplicativos que melhorem a vida do cidadão de Porto Alegre.
