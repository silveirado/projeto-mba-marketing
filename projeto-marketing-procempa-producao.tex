\chapter{Produção}

O foco da produção de conteúdo pela Procempa serão conteúdos relevantes à comunidades de desenvolvedores de sistemas, preferencialmente livres, focados na solução de problemas comunitários e melhorias nos processos de construção de \emph{software}.
Este conteúdo será gerado pelos colaboradores da companhia, levando em consideração as diretrizes fixadas no manual de comunicação, que serão estimulados por campanha de \emph{endomarketing}. Esta campanha será divulgada através da intranet da empresa, e-mail institucional e quadros existentes na companhia.
Serão disponibilizadas três canais de comunicação, que serão o sitio http://github.com/procempa para compartilhamento de código, o sítio http://stackoverflow.com para auxílio na resolução de dúvidas de outros desenvolveres além de blogs pessoais vinculados à companhia, como por exemplo: http://www.procempa.com.br/blogs/derotino.junior.

\section{Compartilhamento de Código Livre}
O sitio GitHub é um repositório de código aberto, utilizado por mais de trinta e oito milhões de projetos ao redor do mundo. Grandes comunidades de software livre tem seus fontes hospedados nesta plataforma.
A criação de conteúdo neste sitio se dará por dois métodos:

\subsection{Criação de Repositórios Procempa}
A criação de repositórios próprios da Procempa será com trechos reutilizáveis de projetos que não interfiram em sigilo da Prefeitura ou que exponham vulnerabilidades de segurança.

Estes projetos tem que ter sentido fora da estrutura de projeto e ter documentação básica necessária para sua utilização. A publicação dos mesmos deverá ser precedida de alinhamento com as divisões de apoio de tecnologia e componentização do departamento de sistemas.

\subsection{Bifurcação de Projetos}
Com a estratégia de apoio e devolução das colaborações que a Procempa recebe através de códigos livre utilizados, os colaboradores devem ser estimulados a bifurcar projetos de software livres que estiverem utilizando no âmbito de suas tarefas para proposição de melhorias e correções.

\section{Resolução de Dúvidas}
O sítio StackOverflow direciona-se a ser um repositório de perguntas e respostas a cerca do desenvolvimento de software nas mais diversas tecnologias. A cada resposta, ou validação de um dada por outro participante, o usuário adquire pontos e medalhas que o estabelecem como referencial em uma tecnologia. Esta estratégia, também conhecida como \emph{gamification}, acaba por gerar grande engajamento dos desenvolvedores. A estratégia é estimular aos colaboradores a criarem perfis profissionais e o utilizarem para interagir com o sitio, criando conteúdos relevantes para a comunidade das tecnologias utilizadas na empresa.

\section{Blogs Pessoais}
Deve-se estimular os colaboradores para que mantenham atualizados blogs pessoais relatando novidades e dificuldades encontradas no dia a dia dos projetos. Compartilhar formas de otimização de processos, sistemas e outros assuntos relativos aos interesses da companhia.

Estes conteúdos devem ser orientados a melhorar o ranqueamento das páginas em mecanismos de busca, como o Google, além de serem referências para pesquisas de cidadãos e empresas interessadas nas tecnologias em uso pela Procempa.
