\section{Públicos-alvo}
Os públicos alvo que trabalharemos neste cenários serão os colaboradores, formadores de opinião na administração municipal, comunidades de software livre e o cidadão de Porto Alegre. Cada um com suas necessidades e anseios perante a tecnologia, assim precisamos equalizar estas percepções para promover a felicidade de forma unanime entre os públicos.

Os colaboradores da Companhia são pessoas de alto nível intelectual, com estabilidade financeira e no trabalho. Em 2015 mais de cem novos colaboradores foram contratados a partir do concurso público realizado no final de 2014, estes são fortemente motivados pelo desafio tecnológico e pela satisfação em suas carreiras pelo reconhecimento público de seu trabalho.

Os formadores de opinião, são pessoas altamente engajadas na gestão pública que buscamo o melhor cenário de aplicação dos recursos e a transparência em suas ações. Estes possuem uma boa capacidade de articulação política para apoiar o atingimento dos objetivos delimitados pelo planejamento de governo e os compromissos públicos assumidos pela administração municipal.

Comunidades de software livre são compostas por pessoas altamente engajadas na defesa da liberdade utilizar, estudar, alterar e melhorar softwares, bem como na garantia do compartilhamento de conhecimento em suas redes. Seus integrantes, via de regra, são muito passionais e críticos à mudanças de estratégias ao mesmo tempo que são árduos colaboradores em projetos que acreditam.

Em foco de todas as estratégias públicas, o cidadão sempre deve ser o maior beneficiado. Seus anseios são pela transparência no trato da coisa pública e por melhores serviços disponíveis. Com a correta aplicação de estratégias resultará em melhores serviços. A abertura de dados, amplamente utilizada na Prefeitura tendo rendido prêmios de transparência em nível nacional, será um incentivo para que iniciativas de novos aplicativos e serviços à população.

Estes diferentes públicos irão requerer diferentes estratégias de divulgação do capital intelectual disponível atualmente na Procempa. Estabelecer-se-á canais e estratégias para cada público e quais informações são relevantes e agregam valor, com isto proporcionar a felicidade de forma igualitária entre todos.
