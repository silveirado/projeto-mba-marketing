\chapter{Metodologia da Pesquisa}

O trabalho será conduzido a partir das referências bibliográficas e com a utilização da metodologia Design thinking, proposta por \citeonline{Brown2010}:
\begin{citacao}
O design thinking evoluiu de um início modesto: artífices como William Morris, arquitetos como Frank Lloyd Wright e \textit{designers} industriais como Henry Dreyfuss e Ray e Charles Eames desejavam tornar o mundo mas acessível, belo e significativo.
\end{citacao}

Assim este trabalho seguirá de forma a buscar incessantemente a inovação e tornar o ecossistema da companhia melhor, com alguns passos propostos por \citeonline{Brown2010}:

\begin{citacao}
\begin{itemize}
\item \textbf{Comece pelo início}: O design thinking começa com a divergência, a tentativa deliberada de expandir a variedade de opções ao invés de restringi-las.
\item \textbf{Assuma uma abordagem centrada no ser humano}: Como o design thinking equilibra as perspectivas do usuário, da tecnologia e dos negócios, é, por natureza, integrador. Como ponto de partida, contudo, ele privilegia o usuário final.
\item \textbf{Fracasse logo, fracasse com frequência}: O tempo até o primeiro protótipo é um bom indicativo da vitalidade de uma cultura de inovação. Com que rapidez as ideias são elaboradas de forma tangível, de modo que possam ser testadas e melhoradas? Os líderes devem incentivar a experimentação e aceitar que não há nada de errado com o fracasso, contanto que ele ocorra no começo e se torne fonte de aprendizado.
\item \textbf{Procure ajuda profissional}: Eu não corto meu próprio cabelo nem troco o óleo do meu carro, embora, provavelmente, seja capaz disto. Em certas ocasiões, faz mais sentido sair de sua organização e buscar oportunidades de expandir o ecossistema de inovação.
\item \textbf{Compartilhe a inspiração}: Não esqueça sua rede interna. Grande parte dos esforços relativos ao compartilhamento de conhecimento ao longo da última década se concentrou na eficiência. Talvez seja a hora de pensar em como suas redes de conhecimento sustentam a \emph{inspiração}.
\end{itemize}
\end{citacao}
Com estes e outros preceitos iremos buscar um trabalho divertido e inovador sempre pensando na companhia como parte de um grande sistema que de acordo com \citeonline{Brown2010} "Aos pensar em função do sistema como um todo as empresas podem se beneficiar de melhores oportunidades" .
