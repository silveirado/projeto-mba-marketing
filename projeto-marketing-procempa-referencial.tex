\chapter{Referencial Teórico}

Vivemos atualmente a era da participação e do marketing colaborativo \cite{Kotler2010}. Com isto as estratégias devem levar em conta o novo consumidor, e cidadão, tecnologicamente incluído e com acesso a uma vasta quantidade informações, assim \citeonline{Kotler2010} afirma:

\begin{citacao}
Desde o início do ano 2000, a tecnologia da informação penetrou o mercado \textit{mainstream}, transformando-se no que consideramos hoje a nova onda de tecnologia. [...] A tecnologia permite que os indivíduos se expressem e colaborem entre si. [...] Na era da participação, as pessoas criam e consomem notícias, ideias e entretenimento. A nova onda de tecnologia transforma as pessoas de consumidores em prosumidores.
\end{citacao}

Com este embasamento propor-se-á uma estratégia de envolvimento dos consumidores - sejam eles cidadãos, funcionários, participantes de comunidades de \textit{software} de código aberto ou outras empresas, sejam elas públicas ou privadas - para que tornem-se os prosumidores propostos por \citeonline{Kotler2010}, que ajudam a construir a imagem e o produto da empresa. Este consumidor é definido por \citeonline{Torres2009} como:

\begin{citacao}
O consumidor on-line é a mesma pessoa, de carne e osso, que está na via real lendo uma revista ou assistindo televisão. Mas quando ele entra na Internet, quando ele está on-line, surgem comportamentos que muitas vezes ele não apresentava na vida real por estar limitado pelas restrições de tempo, espaço ou dinheiro.
\end{citacao}

Para estes novos consumidores o primeiro passo de uma boa estratégia, segundo \citeonline{Torres2009}, é definir o que ele quer:

\begin{citacao}
Em primeiro lugar, pense que [...] o consumidor não está na Internet para fazer "a mesma coisa" que fazia em outra mídia.[...] Em segundo lugar, é necessário lembrar que o consumidor está fazendo conectado junto ao seu computador.[...] O consumidor quando está conectado a internet tem basicamente três desejos, três necessidades que são como um farol-guia e nos ajudam a entender seu comportamento. Estas três necessidades, criadas e satisfeitas pelo próprio consumidor, são: informação, diversão e relacionamento.
\end{citacao}

Tendo em vista o foco neste novo consumidor utilizaremos como base do plano a metodologia dos 8 Ps propostas por \citeonline{Adolpho2011}:

\begin{citacao}
A metodologia dos 8 Ps faz com que a empresa mantenha o foco no método, no conceito, no que deve ser feito. As ferramentas que serão utilizadas para isso podem ser quaisquer que sirvam às definições da empresa a respeito de sua estratégia digital. Hoje pode ser o Twitter, amanhã pode ser qualquer outra que apareça com o mesmo propósito da comunicação imediata, assim como, para determinadas pessoas, ontem era o Orkut e hoje é o Facebook. As ferramentas mudam a todo momento. A cada semana aparece alguma nova funcionalidade, um novo site. Repito: a ferramenta, se é Twitter ou Facebook, não é importante. O importante é entender o conceito de cada P e, ao chegar a hora dele, olhar para o mercado e ver o que há disponível que mais se adapte a sua meta.
\end{citacao}

\textbf{Primeiro P: Pesquisa}, neste P estabeleceremos uma pesquisa para definir o cliente o ambiente da empresa e quais dados são relevantes para alavancarmos as ações da empresa, tem em mente o que \citeonline{Adolpho2011} preconiza:
\begin{citacao}
"O consumidor do novo século baseado na tecnologia da informação é muito mais ativo porque tem as ferramentas para tal. [...] À medida que ele faz tudo isso, ele se expõe. Ele deixa muitos rastro. A grande diferença da internet para todos os outros ambientes é que na internet tudo pode ser medido. Esses rastros podem ser lidos e pesquisados."
\end{citacao}

\textbf{Segundo P: Planejamento}, nesta fase será elaborado um conjunto detalhado de ações a partir dos dados coletados na primeira etapa, conforme \citeonline{Adolpho2011}:

\begin{citacao}
De posse das informações descobertas no 1º P, do conhecimento de como age o consumidor no meio online, do que os outros membros da equipe fazem e sabendo de forma e sabendo de forma clara qual a missão crítica do site, parte-se para elaborar um documento que será a diretriz de todo o projeto. A equipe fará o planejamento de marketing digital que será seguido até sua conclusão.
\end{citacao}

Neste ponto iremos tratar dos valores da empresa e as diretrizes, segundo as quais o trabalho será pautado. \citeonline{Kotler2010} estabelecem dos tipos de valores, os compartilhados, relativos a marca, e o comportamento usual dos empregados. Segundo o mesmo:

\begin{citacao}
Desenvolver uma cultura corporativa significa alinhar os valores compartilhados com o comportamento usual. Em outras palavras, significa demonstrar os valores no comportamento no dia a dia da empresa. A combinação de valores e comportamento dos empregados deve refletir a missão da marca da empresa. É importante que os empregados ajam como embaixadores dos valores para transmitir a missão da marca aos consumidores.
\end{citacao}

\textbf{Terceiro P: Produção}, fase de preparar a tecnologia de compartilhamento de conteúdo para suportar as estratégias e ações definidas no 2º P de \citeonline{Adolpho2011}.

\begin{citacao}
Nesse "P" você irá otimizar o código de seu site para mecanismo de busca de modo a se posicionar nas primeiras colocações dos resultados do Google, e com isso
atrair o enorme tráfego de consumidores que pesquisam no Google. O 3º P se concentra na estrutura do site, em suas funcionalidades. [...] Em suma, no 3º P você aprenderá a ter um site pronto para ser uma plataforma de negócios, porém, ainda será um carro de Fórmula 1 sem o que faz ele de fato
funcionar - a gasolina.
\end{citacao}

\textbf{Quarto P: Publicação}, neste P iremos criar o conteúdo para divulgação das ações da companhia nas suas comunidades. Segundo \citeonline{Adolpho2011}:

\begin{citacao}
O site para que tenha uma taxa de conversão alta, deve ser relevante para o público-alvo e [...] relevância se constrói com conteúdo. A percepção
das pessoas com relação a sua empresa será construída pelo conteúdo que ela apresenta a respeito das próprias pessoas. Os consumidores querem ver algo que diga respeito a eles, não à empresa.
\end{citacao}

Iniciaremos com a criação do conteúdo de marketing da missão da empresa, para que seja possível atrair o consumidor, segundo \citeonline{Kotler2010}:

\begin{citacao}
Para fazer o marketing da missão da empresa ou do produto junto aos consumidores, as empresas precisam oferecer uma missão de transformação, criar histórias atrativas em torno dela e envolver os consumidores em sua concretização.
\end{citacao}

Além do institucional, partiremos para a criação de blogs pessoais dos colaboradores, seguindo as regras e valores da companhia, para que possam expressar suas criações e trabalhos, baseando-se no relato de \citeonline{Kotler2010}:

\begin{citacao}
A popularidade dos blogs e do Twitter chegou ao mundo corporativo. A IBM, por exemplo, estimula seus funcionários a criar blogs específicos em que podem falar livremente sobre a empresa, desde que sigam determinadas diretrizes. Outro exemplo é a General Eletric, que criou o Tweet Squad, um grupo de jovens empregados que treinam empregados mais velhos, ensinado-os a usar mídias sociais.
\end{citacao}

Este ponto é defendido por \citeonline{Kotler2010}:

\begin{citacao}
Diz um provérbio chinês: \emph{"Conta-me e eu esquecerei, mostra-me e talvez eu me lembre; envolva-me e eu entenderei."} Isto é relevante para o \textit{empowerment} do funcionário. Os empregados precisam estar envolvidos e precisam ter autonomia para agir. Os valores da empresa mudaram sua vida. Agora, é sua vez de mudar a vida dos outros. Trata-se de criar uma plataforma a partir da qual os empregados possam fazer a diferença.
\end{citacao}

Unindo estas duas comunidades, a externa a companhia e os colaboradores, obteremos os melhores resultados, de acordo com \citeonline{Kotler2010}: "O marketing de seus valores junto aos empregados é tão importante quanto o marketing da missão junto aos consumidores."  Esta será a orientação para os conteúdos a serem gerados, sempre de forma colaborativa e inovadora, conforme  \citeonline{Kotler2010}: "A colaboração também pode ser a nova fonte de inovação."

\textbf{Quinto P: Promoção} é fase de executar as ações planejadas para a divulgação dos conteúdos da companhia, usando os consumidores como veículos, de acordo com \citeonline{Kotler2010}:

\begin{citacao}
A estratégia de transformar o consumidor em veículo é uma das mais eficazes que se tem em termos de resultados finais de vendas e construção de marca, porém
não é um trabalho que traga ação rápida. O tempo de maturação pode ser de meses para que a marca se espalhe. É lógico que isso depende da força que a sua marca
já tem no mercado, da relevância da campanha para o público-alvo, da qualidade do conteúdo ao qual ela remete e vários outros fatores que farão com que sua
campanha de web 2.0 dê resultados mais lentamente ou não.
\end{citacao}

Para isto devemos criar materiais de propaganda, focados em cada canal, fortalecendo a marca e fomentando a criação de relacionamento com a companhia.


\textbf{Sexto P: Propagação}
No atual cenário de internet a simples propaganda não mais tem o mesmo efeito, assim é necessário que a propagação do boca a boca leve nosso conteúdo, até atinja massa crítica suficiente para torna-se viral e possa atingir o maior número de pessoas, mesmo sem novos esforços. Para \citeonline{Kotler2010}:

\begin{citacao}
Será no 6º P, por meio da comunicação pessoa a pessoa, que sua marca ganhará reputação (palavra essencial na nossa economia da transparência). É na comunicação viral feita pelos consumidores que sua marca chegará aos recônditos do mercado e atingirá de forma muito mais barata, lucrativa, eficiente e confiável consumidores de todos os segmentos possíveis. A propagação é a chave do marketing viral [...]
\end{citacao}

Para fins de permitir este compartilhamento e propagação dos conteúdos serão baseadas nos estudos de \citeonline{Malcolm2002} sobre a propagação de ideias como sendo epidemias. Segundo o mesmo, em algumas condições, há um ponto de desequilíbrio em que pequenas ações se tornam uma grande epidemia, propagando o conteúdo como se fosse um vírus. Para isto existem regras, segundo \citeonline{Malcolm2002}:"As três regras do Ponto de Desequilíbrio - a Regra dos Eleitos, o Fator de Fixação, o Poder do Contexto".
Com isto devemos direcionar os conteúdos para:

\begin{itemize}
\item \emph{Regra dos eleitos}: Focar a divulgação inicial em pessoas com potencialidade de disseminação e que sejam referências nas áreas específicas.
\item \emph{Fator de fixação}: Os conteúdos tem que ser relevantes e tratados de forma adequada a cada canal. Além de ter componentes que facilitem a fixação do seu conteúdo.
\item \emph{Poder do contexto}: Sejam publicadas no contexto correto, ou seja, se técnica dentro dos congressos, se melhorias para o cidadão nas fretes do Orçamento Participativo. Neste ponto devemos concentrar os esforços de divulgação onde o público alvo se encontra.
\end{itemize}

\textbf{Sétimo P: Personalização}
Os conteúdos devem ser personalizados, pois os prosumidores \cite{Kotler2010} assim o exigem e segundo \citeonline{Kotler2010}:

\begin{citacao}
O seu site, porém, pode ter uma excelente memória se for programado para isso (e planejado lá no 2º P). Para se relacionar com as pessoas você adapta o seu comportamento de acordo com a pessoa com a qual está falando. A sua marca, para se relacionar na internet, deve fazer o mesmo. A personalização deve passar   por todas as etapas da sua ação de internet. Desde a navegação do usuário até o e-mail que  envia para ele. Segmentar o mercado é funcional para isso - o que veremos.  O ideal é que você faça uma microssegmentação no âmbito do consumidor.
\end{citacao}

Para permitir a personalização deveremos segmentar as publicações e direcioná-las a cada mídia e público, além de permitir ao consumidor que selecione os conteúdos que quer, a frequência e como o quer. Este ponto é de fundamental importância, pois segundo \citeonline{Torres2009} toda a interação deve ser consentida:

\begin{citacao}
Talvez a regra de ouro do uso da Internet é que qualquer interação deve ser consentida. Embora os piratas da Internet, os spammers e hackers, tenho mostrado caminhos duvidosos para ganhar dinheiro, quando falamos em marketing e publicidade e, portanto, em marcas e empresas reais, não podemos cair na tentação de invadir o ambiente do usuário sem consentimento.
\end{citacao}

\textbf{Oitavo P: Precisão}: Para um efetivo retorno e avaliação de desempenho das estratégias aplicadas, faz-se necessária a criação de indicadores que sirvam tanto para este fim quanto para projetar novas ações. \citeonline{Kotler2010} coloca que:

\begin{citacao}
Não medir os resultados obtidos depois de uma ação é andar cego a 180 km/h em uma estrada cheia de curvas. [...] A chance de você fazer algo errado é enorme. Não faz mais sentido não mensurar resultados, uma vez que agora as empresa têm a possibilidade de fazê-lo. [...] Só assim você poderá fortalecer o que deu certo e eliminar o que não deu, aumentando sua margem de acerto ao longo do tempo.
\end{citacao}

Nesta fase serão criado e implantados indicadores estratégicos e operacionais da efetividade do que está sendo aplicado, gerando planos de ação para cada um dos mesmos.
