\chapter{Referencial Teórico}

Vivemos atualmente a era da participação e do marketing colaborativo \cite{Kotler2010}. Com isto as estratégias devem levar em conta o novo consumidor, e cidadão, tecnologicamente incluído e com acesso a uma vasta quantidade informações, assim \citeonline{Kotler2010} afirma:
\begin{citacao}
Desde o início do ano 2000, a tecnologia da informação penetrou o mercado \textit{mainstream}, transformando-se no que consideramos hoje a nova onda de tecnologia. [...] A tecnologia permite que os indivíduos se expressem e colaborem entre si. [...] Na era da participação, as pessoas criam e consomem notícias, ideias e entretenimento. A nova onda de tecnologia transforma as pessoas de consumidores em prosumidores.
\end{citacao}

Com este embasamento propor-se-á uma estratégia de envolvimento dos consumidores - sejam eles cidadãos, funcionários, participantes de comunidades de \textit{software} de código aberto ou outras empresas, sejam elas públicas ou privadas - para que tornem-se os prosumidores propostos por \citeonline{Kotler2010}, que ajudam a construir a imagem e o produto da empresa. Este consumidor é definido por \citeonline{Torres2009} como:
\begin{citacao}
O consumidor on-line é a mesma pessoa, de carne e osso, que está na via real lendo uma revista ou assistindo televisão. Mas quando ele entra na Internet, quando ele está on-line, surgem comportamentos que mutas vezes ele não apresentava na vida real por estar limitado pelas restrições de tempo, espaço ou dinheiro.
\end{citacao}
