\chapter{Conclusão}

O momento indica que desenvolver software passa a ser um exercício de criação, muito mais do que técnica, e a computação passa a ser um meio e não o fim. O computador e todo seu poder de processamento, passa a ser ferramenta, assim como o cinzel na mão de Michelangelo. Para que a mente tenha esta capacidade de criação é fundamental que a liberdade e a responsabilidade sejam cláusulas pétreas nas equipes e a comunicação e acesso a informação estejam no centro das estratégias.

A proposição para este cenário é o estudo da arte de desenvolver aplicações de alto valor agregado, escaláveis e confiáveis, sendo feliz durante sua construção. Para isto precisamos colocar os valores humanos sempre em primeiro lugar, investir fortemente em formação técnica, tendo em mente que essa é apenas o meio nunca o fim, buscar qualificação emocional e, principalmente, buscar o  que nos faz feliz. Sabe-se que as dificuldades do caminho não são pequenas, mas a persistência, foco e treinamento nos levarão a construir nosso próprio David, trazendo à vida aplicações úteis e que possam agregar valor à sociedade e satisfação ao seu criador.

Neste planejamento buscamos demonstrar como o empoderamento dos colaboradores pode melhorar sua produtividade e alavancar a Companhia ao patamar de referência em construção de software e processos. A base de tudo o que precisamos para isto já está disponível em larga quantidade de patrimônio intelectual gerado a cada dia pelas equipes. Assim somente precisamos apor os trilhos para que essa grande locomotiva de conhecimento possa acelerar resultados tangíveis para todos, sejam eles a Prefeitura, os cidadãos ou as comunidades de software.
