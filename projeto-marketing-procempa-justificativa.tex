\chapter{Justificativa}

A Companhia de Processamento de Dados do Município de Porto Alegre  - Procempa - foi fundada em 9 de setembro de 1977 e sempre foi reconhecida nacionalmente como referência em tecnologia das cidades. Em vários momentos a empresa ganhou renomados prêmios do setor tecnológico e foi expoente em tecnologias de ponta.
No ano de 2014 a empresa definiu uma estratégia de crescimento, juntamente com a administração municipal, planejando e colocando em prática diversas ações que a colocaram em situação de prontidão para o atual momento e para as crescentes inovações do mercado. As principais ações foram:

\begin{itemize}
\item Execução de concurso público com a contratação de mais de 100 novos funcionários;
\item Estruturação de processos internos de controle e administração, visando melhorias na gestão e transparência;
\item Estabelecimento de gestão por metas e indicadores;
\item Atualização das estruturas de suporte aos sistemas e servidores;
\item Reforma das instalações visando a agilidade e conforto dos funcionários e
\item Contratação de consultoria para apoiar a implantação de métodos ágeis.
\end{itemize}
Este cenário desenha-se perfeito para o início de uma estratégia de marketing para apoio a esta grande evolução e divulgação dos resultados obtidos com as estratégias. O foco destas estratégias será o fomento de soluções que possam ajudar a melhorar a vida do cidadão, seja fornecendo subsídios, como modelos, tecnologias, metodologias, etc, para novas \textit{startups} no município, seja fornecendo novos serviços diretamente aos cidadãos, para melhorar o seu dia a dia.
A proposta do tema, tendo a felicidade como foco, é trazer um excelente ambiente de desenvolvimento e prestação de serviços à comunidade. Neste planejamento a satisfação dos analistas, gestores, prefeitura e comunidade serão levadas em conta de forma equânime.
Estas estratégias irão se basear em mídias sociais, estratégias de divulgação de informações, incentivo a produção de conteúdo pelos colaboradores, participação em comunidades livres, fornecendo e recebendo apoio das mesmas.
